\documentclass[a4paper,12pt]{article}
    \author{
        Randy Henry \\ 
        \href{mailto:pivotsallit@gmail.com}
        {{\small \texttt{pivotsallit@gmail.com}}}}
    \title{Lang Syntax Reference}

    \usepackage{url}
    \usepackage{hyperref}
    \usepackage{listings}

    \lstset{
        language=Haskell,
        basicstyle=\footnotesize\ttfamily,
        frame=single,
        keepspaces=true,
        tabsize=4
    }


    % \usepackage{syntonly}
    % \syntaxonly % for debugging; comment this out

\begin{document}
    \maketitle
    \tableofcontents

    \newpage

    \section{Introduction}
        This paper walks through the "Syntax Across Languages" \cite{SAL}
        page, describing how each phenomenon it references would be translated
        into valid \textsc{Lang} syntax.

        This is meant not as an introductory guide to \textsc{Lang}, but as a sort 
        of formal reference for how common design patterns would appear 
        in \textsc{Lang} code out in the wild.

    \section{Miscellaneous} 
        \subsection{Commenting}
            \begin{center}
                \begin{tabular*}{\textwidth}{ l@{\extracolsep{\fill}}l }
                    Until end of line & \lstinline|-- ...| \\ [0.5em]
                    \hline \\ [-0.5em]
                    Nestable          & \lstinline$--/ ... /--$ \\ [0.5em]
                \end{tabular*}
            \end{center}

            \subsubsection{Documentation comment}
                \begin{center}
                    \begin{tabular*}{\textwidth}{ l@{\extracolsep{\fill}}l }
                        Until end of line & \lstinline$--| ...$ 
                        \\ [0.5em] \hline \\ [-0.5em]
                        Nestable          & \lstinline$--| ... |--$
                        \\ [0.5em]
                    \end{tabular*}
                \end{center}

        \subsection{Tokens}
            \begin{center}
                \begin{tabular*}{\textwidth}{ l@{\extracolsep{\fill}}l }
                    Case-sensitive       & \lstinline|x =/= X.| 
                    \\ [0.5em] \hline \\ [-0.5em]
                    kebab-case-variables & \lstinline|avogadro's-number := 6.022e23.|
                    \\ [0.5em] \hline \\ [-0.5em]
                    Upper-Kebab-Types    & \lstinline|type 2D-Point := { x: Z, y: Z. }.|
                    \\ [0.5em] \hline \\ [-0.5em]
                    Identifier regexp    & 
                        \lstinline|[_a-zA-Z!0&*/:<=>?^][_a-zA-Z!0&*/:<=>?^0-9.+-]*|
                \end{tabular*}
            \end{center}
            


% \paragraph{Until end of line}
% \verb|-- ...|
% \paragraph{Nestable}
% \verb|(* ... *)|
       
    % TODO: add more stuff

    \newpage
    \bibliographystyle{apalike}
    \bibliography{syntax}
\end{document}
