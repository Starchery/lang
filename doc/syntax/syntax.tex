\documentclass[a4paper,12pt]{article}
    \author{
        Randy Henry \\ 
        \href{mailto:pivotsallit@gmail.com}
        {{\small \texttt{pivotsallit@gmail.com}}}}
    \title{Lang Syntax Reference}

    % \usepackage{syntonly}
    % \syntaxonly % for debugging; comment this out

    \usepackage{url}
    \usepackage{hyperref}
    \usepackage{listings}

    \lstset{
        language=Haskell,
        basicstyle=\footnotesize\ttfamily,
        frame=single,
        keepspaces=true,
        tabsize=4
    }

    \newcommand{\code}{\lstinline}
    
    \newcommand{\br}{\\ [0.5em] \hline \\ [-0.5em]}

    \newenvironment{data}
    {
        \begin{center}
        \begin{tabular*}{\textwidth}{ l@{\extracolsep{\fill}}l }
    }
    {
        \end{tabular*}
        \end{center}
    }

%----------------------------------------------------------------------------%
%----------------------------------------------------------------------------%
%----------------------------------------------------------------------------%

\begin{document}
    \maketitle
    \tableofcontents

    \newpage

    \section{Introduction}
        This paper walks through the "Syntax Across Languages" \cite{SAL}
        page, describing how each phenomenon it references would be translated
        into valid \textsc{Lang} syntax.

        This is meant not as an introductory guide to \textsc{Lang}, but as a sort 
        of formal reference for how common design patterns would appear 
        in \textsc{Lang} code out in the wild.

    \section{Miscellaneous} 
        \subsection{Commenting}
            \begin{data}
                Until end of line & 
                    \code|-- This is a comment.| \br
                Nestable          & 
                    \code$--/ This is a comment. /--$
            \end{data}

            \subsubsection{Documentation comment}
                \begin{data}
                    Until end of line & 
                        \code|-- This is a comment.| \br
                    Nestable          & 
                        \code$--| This is a comment. |--$
                \end{data}

        \subsection{Tokens}
            \begin{data}
                Case-sensitive       & 
                    \code|x =/= X.| \br
                kebab-case-variables & 
                    \code|avogadro's-number := 6.022e23.| \br
                Upper-Kebab-Types    & 
                    \code|type 2D-Point := { x: Z, y: Z. }.| \br
                Identifier regexp    & 
                    \code|[_a-zA-Z!0&*/:<=>?^][_a-zA-Z!0&*/:<=>?^0-9.+-]*|
            \end{data}
            
        \subsection{Variable assignment/declaration}
            \begin{data}
                Declaration        & 
                    \code|x := 3.|             \br
                Assignment         & 
                    \code|x :- 3. x :- x + 2.| \br
                Scoped Declaration & 
                    \code|let x := expr in { ... }|
            \end{data}

    % TODO: add more stuff

    \newpage
    \bibliographystyle{apalike}
    \bibliography{syntax}
\end{document}
